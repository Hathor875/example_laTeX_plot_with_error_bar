\documentclass{article}
\usepackage{pgfplots}
\usepackage{pgfplotstable}
\usepackage[T1]{fontenc}
\pgfplotsset{compat=1.16}
\pagestyle{empty}
\begin{document}
\begin{tikzpicture}
\hspace*{-5cm} % Przesunięcie w lewo o 2 cm (możesz dostosować)
\begin{axis}[
 trim axis left,
  title={Wykres zależności I(U) Schemat A dla żarówki},
  xlabel={Napięcie [V]},
  ylabel={Natężenie [A]},
  grid=major,
  width=1.6\textwidth,
  height=1.6\textwidth,
  enlargelimits=false,
  scaled y ticks=false,
  xmin=0,
  xmax=25,
  ymin=0.2,
  ymax=1,
  mark size=1, % Rozmiar punktów pomiarowych
  only marks,  % Tylko punkty, bez linii
]

\addplot+ [
  error bars/.cd,
  x dir=both,
  x explicit,
  y dir=both,
  y explicit,
] table [
  x=x,
  y=y,
  x error plus=ex,
  x error minus=ex,
  y error plus=ey,
  y error minus=ey,
] {
 x y ey   ex
1.97    0.30    0.056   0.01985
3.97    0.38    0.0576  0.02485
5.94    0.45    0.059   0.0397
7.96    0.52    0.0604  0.0498
9.92    0.58    0.0616  0.0596
11.93   0.63    0.0626  0.06965
13.89   0.68    0.0636  0.07945
15.92   0.72    0.0644  0.0896
17.86   0.77    0.0654  0.0993
19.86   0.81    0.0662  0.1093
21.8    0.86    0.0672  0.209
23.8    0.90    0.068   0.219
};

% Linia aproksymacji (czarna, przerywana)
\addplot [smooth, black, dashed] table [
  x=x,
  y={create col/linear regression={y=y}}
] {
x y ey   ex
1.97    0.30    0.056   0.01985
3.97    0.38    0.0576  0.02485
5.94    0.45    0.059   0.0397
7.96    0.52    0.0604  0.0498
9.92    0.58    0.0616  0.0596
11.93   0.63    0.0626  0.06965
13.89   0.68    0.0636  0.07945
15.92   0.72    0.0644  0.0896
17.86   0.77    0.0654  0.0993
19.86   0.81    0.0662  0.1093
21.8    0.86    0.0672  0.209
23.8    0.90    0.068   0.219
};

\end{axis}
\end{tikzpicture}




\begin{tikzpicture}
\hspace*{-5cm} % Przesunięcie w lewo o 2 cm (możesz dostosować)
\begin{axis}[
  title={Wykres zależności I(U) Schemat B dla żarówki},
  xlabel={Napięcie [V]},
  ylabel={Natężenie [A]},
  grid=major,
  width=1.6\textwidth,
  height=1.6\textwidth,
  enlargelimits=false,
  scaled y ticks=false,
  xmin=0,
  xmax=25,
  ymin=0.2,
  ymax=1,
  mark size=1, % Rozmiar punktów pomiarowych
  only marks,  % Tylko punkty, bez linii
]

\addplot+ [
  error bars/.cd,
  x dir=both,
  x explicit,
  y dir=both,
  y explicit,
] table [
  x=x,
  y=y,
  x error plus=ex,
  x error minus=ex,
  y error plus=ey,
  y error minus=ey,
] {
x y ey   ex
1.97    0.27    0.0554  0.01985
3.97    0.36    0.0572  0.02985
5.99    0.44    0.0588  0.03995
7.91    0.5     0.06    0.04955
9.94    0.56    0.0612  0.0597
11.88   0.62    0.0624  0.0694
13.87   0.67    0.0634  0.07935
15.89   0.72    0.0644  0.08945
17.91   0.76    0.0652  0.09955
19.83   0.81    0.0662  0.10915
21.8    0.85    0.067   0.209
23.8    0.89    0.0678  0.219
};

% Linia aproksymacji (czarna, przerywana)
\addplot [smooth, black, dashed] table [
  x=x,
  y={create col/linear regression={y=y}}
] {
x y ey   ex
1.97    0.27    0.0554  0.01985
3.97    0.36    0.0572  0.02985
5.99    0.44    0.0588  0.03995
7.91    0.5     0.06    0.04955
9.94    0.56    0.0612  0.0597
11.88   0.62    0.0624  0.0694
13.87   0.67    0.0634  0.07935
15.89   0.72    0.0644  0.08945
17.91   0.76    0.0652  0.09955
19.83   0.81    0.0662  0.10915
21.8    0.85    0.067   0.209
23.8    0.89    0.0678  0.219
};

\end{axis}
\end{tikzpicture}


\begin{tikzpicture}
\hspace*{-5cm} % Przesunięcie w lewo o 2 cm (możesz dostosować)
\begin{axis}[
  title={Wykres zależności I(U) Schemat A dla drutu oporowego},
  xlabel={Napięcie [V]},
  ylabel={Natężenie [mA]},
  grid=major,
  width=1.7\textwidth,
  height=1.7\textwidth,
  enlargelimits=false,
  scaled y ticks=false,
  xmin=1,
  xmax=25,
  ymin=0,
  ymax=260,
  mark size=1, % Rozmiar punktów pomiarowych
  only marks,  % Tylko punkty, bez linii
]

\addplot+ [
  error bars/.cd,
  x dir=both,
  x explicit,
  y dir=both,
  y explicit,
] table [
  x=x,
  y=y,
  x error plus=ex,
  x error minus=ex,
  y error plus=ey,
  y error minus=ey,
] {
x y ey   ex
1.99    16.9  0.3028    0.01995
3.96    33.7  0.5044    0.0298
5.96    50.7  0.7084    0.0398
7.88    67.1  0.9052    0.0494
9.85    84   1.108     0.05925
11.86   101.4 1.3168    0.0693
13.77   118.1  1.5172    0.07885
15.73   135.2  1.7224    0.08865
17.72   150.6  1.9072    0.0986
19.67   169.6  2.1352    0.10835
21.6    187   2.344     0.208
23.8    200     54       0.219
};

% Linia aproksymacji (czarna, przerywana)
\addplot [smooth, black, dashed] table [
  x=x,
  y={create col/linear regression={y=y}}
] {
x y ey   ex
1.99    16.9  0.3028    0.01995
3.96    33.7  0.5044    0.0298
5.96    50.7  0.7084    0.0398
7.88    67.1  0.9052    0.0494
9.85    84   1.108     0.05925
11.86   101.4 1.3168    0.0693
13.77   118.1  1.5172    0.07885
15.73   135.2  1.7224    0.08865
17.72   150.6  1.9072    0.0986
19.67   169.6  2.1352    0.10835
21.6    187   2.344     0.208
23.8    200     54       0.219
};

\end{axis}
\end{tikzpicture}

\begin{tikzpicture}
\hspace*{-5cm} % Przesunięcie w lewo o 2 cm (możesz dostosować)
\begin{axis}[
  title={Wykres zależności I(U) Schemat B dla drutu oporowego},
  xlabel={Napięcie [V]},
  ylabel={Natężenie [mA]},
  grid=major,
  width=1.7\textwidth,
  height=1.7\textwidth,
  enlargelimits=false,
  scaled y ticks=false,
  xmin=0,
  xmax=25,
  ymin=10,
  ymax=260,
  mark size=1, % Rozmiar punktów pomiarowych
  only marks,  % Tylko punkty, bez linii
]

\addplot+ [
  error bars/.cd,
  x dir=both,
  x explicit,
  y dir=both,
  y explicit,
] table [
  x=x,
  y=y,
  x error plus=ex,
  x error minus=ex,
  y error plus=ey,
  y error minus=ey,
] {
x y ey   ex
1.99    17.4  0.3088    0.01995
3.95    34.6  0.5152    0.02975
5.9     51.6  0.7192    0.0395
7.87    68.8  0.9256    0.04935
9.81    85.8  1.1296    0.05905
11.8    103.2  1.3384    0.069
13.76   120.4  1.5448    0.0788
15.75   137.9  1.7548    0.08875
17.66   154.8  1.9576    0.0983
19.64   172.3  2.1676    0.1082
21.6    189.8  2.3776    0.208
23.8    200    54       0.219
};

% Linia aproksymacji (czarna, przerywana)
\addplot [smooth, black, dashed] table [
  x=x,
  y={create col/linear regression={y=y}}
]  {
x y ey   ex
1.99    17.4  0.3088    0.01995
3.95    34.6  0.5152    0.02975
5.9     51.6  0.7192    0.0395
7.87    68.8  0.9256    0.04935
9.81    85.8  1.1296    0.05905
11.8    103.2  1.3384    0.069
13.76   120.4  1.5448    0.0788
15.75   137.9  1.7548    0.08875
17.66   154.8  1.9576    0.0983
19.64   172.3  2.1676    0.1082
21.6    189.8  2.3776    0.208
23.8    200    54       0.219
};

\end{axis}
\end{tikzpicture}




\begin{figure}
\centering

\begin{tikzpicture}
\begin{axis}[
  title={Porównanie wykresów Schemat A i B dla żarówki},
  xlabel={Napięcie [V]},
  ylabel={Natężenie [A]},
  grid=major,
  width=0.8\textwidth,
  height=0.8\textwidth,
  enlargelimits=false,
  scaled y ticks=false,
  xmin=0,
  xmax=25,
  ymin=0.2,
  ymax=1,
  mark size=1, % Rozmiar punktów pomiarowych
  only marks,  % Tylko punkty, bez linii
]

% Linie dla Schematu A
\addplot+ [
  error bars/.cd,
  x dir=both,
  x explicit,
  y dir=both,
  y explicit,
] table [
  x=x,
  y=y,
  x error plus=ex,
  x error minus=ex,
  y error plus=ey,
  y error minus=ey,
] {
  x y ey   ex
  1.97    0.30    0.056   0.01985
  3.97    0.38    0.0576  0.02485
  5.94    0.45    0.059   0.0397
  7.96    0.52    0.0604  0.0498
  9.92    0.58    0.0616  0.0596
  11.93   0.63    0.0626  0.06965
  13.89   0.68    0.0636  0.07945
  15.92   0.72    0.0644  0.0896
  17.86   0.77    0.0654  0.0993
  19.86   0.81    0.0662  0.1093
  21.8    0.86    0.0672  0.209
  23.8    0.90    0.068   0.219
};

% Linie dla Schematu B
\addplot+ [
  error bars/.cd,
  x dir=both,
  x explicit,
  y dir=both,
  y explicit,
] table [
  x=x,
  y=y,
  x error plus=ex,
  x error minus=ex,
  y error plus=ey,
  y error minus=ey,
] {
  x y ey   ex
  1.97    0.27    0.0554  0.01985
  3.97    0.36    0.0572  0.02985
  5.99    0.44    0.0588  0.03995
  7.91    0.5     0.06    0.04955
  9.94    0.56    0.0612  0.0597
  11.88   0.62    0.0624  0.0694
  13.87   0.67    0.0634  0.07935
  15.89   0.72    0.0644  0.08945
  17.91   0.76    0.0652  0.09955
  19.83   0.81    0.0662  0.10915
  21.8    0.85    0.067   0.209
  23.8    0.89    0.0678  0.219
};

\end{axis}
\end{tikzpicture}

\begin{tikzpicture}
\begin{axis}[
  title={Porównanie wykresów Schemat A i B dla drutu oporowego},
  xlabel={Napięcie [V]},
  ylabel={Natężenie [mA]},
  grid=major,
  width=0.5\textwidth,
  height=0.5\textwidth,
  enlargelimits=false,
  scaled y ticks=false,
  xmin=1,
  xmax=25,
  ymin=0,
  ymax=260,
  mark size=1, % Rozmiar punktów pomiarowych
  only marks,  % Tylko punkty, bez linii
]

% Linie dla Schematu A (dla drutu oporowego)
\addplot+ [
  error bars/.cd,
  x dir=both,
  x explicit,
  y dir=both,
  y explicit,
] table [
  x=x,
  y=y,
  x error plus=ex,
  x error minus=ex,
  y error plus=ey,
  y error minus=ey,
] {
  x y ey   ex
  1.99    16.9  0.3028    0.01995
  3.96    33.7  0.5044    0.0298
  5.96    50.7  0.7084    0.0398
  7.88    67.1  0.9052    0.0494
  9.85    84   1.108     0.05925
  11.86   101.4 1.3168    0.0693
  13.77   118.1  1.5172    0.07885
  15.73   135.2  1.7224    0.08865
  17.72   150.6  1.9072    0.0986
  19.67   169.6  2.1352    0.10835
  21.6    187   2.344     0.208
  23.8    200     54       0.219
};

% Linie dla Schematu B (dla drutu oporowego)
\addplot+ [
  error bars/.cd,
  x dir=both,
  x explicit,
  y dir=both,
  y explicit,
] table [
  x=x,
  y=y,
  x error plus=ex,
  x error minus=ex,
  y error plus=ey,
  y error minus=ey,
] {
  x y ey   ex
  1.99    17.4  0.3088    0.01995
  3.95    34.6  0.5152    0.02975
  5.9     51.6  0.7192    0.0395
  7.87    68.8  0.9256    0.04935
  9.81    85.8  1.1296    0.05905
  11.8    103.2  1.3384    0.069
  13.76   120.4  1.5448    0.0788
  15.75   137.9  1.7548    0.08875
  17.66   154.8  1.9576    0.0983
  19.64   172.3  2.1676    0.1082
  21.6    189.8  2.3776    0.208
  23.8    200    54       0.219
};

\end{axis}
\end{tikzpicture}

\caption{Porównanie wykresów dla różnych obwodów pomiarowych }
\end{figure}

\begin{tikzpicture}
\begin{axis}[
  title={Zależność R(I) dla żarówki },
  ylabel={Rezystancja [R]},
  xlabel={Natężenie [A]},
  grid=major,
  width=1\textwidth,
  height=1\textwidth,
  enlargelimits=false,
  scaled y ticks=false,
  xmin=0.2,
  xmax=1,
  ymin=4,
  ymax=30,
  mark size=1, % Rozmiar punktów pomiarowych
  only marks,  % Tylko punkty, bez linii
]

% Linie dla Schematu A (dla drutu oporowego)
\addplot+ [
  error bars/.cd,
  x dir=both,
  x explicit,
  y dir=both,
  y explicit,
] table [
  x=x,
  y=y,
  x error plus=ex,
  x error minus=ex,
  y error plus=ey,
  y error minus=ey,
] {
  y  x ey   ex
  6.56    0.27  1.6    0.0554
  10.47    0.36  1.9    0.0572
  13.2    0.44     2.0 0.0588
  15.30    0.5   2.0  0.06
  17.1   0.56     2.1  0.0612
  18.93   0.62  2.1    0.0624
  20.42   0.67  2.1    0.0634
  22.1   0.72   2.1 0.0644
  23.19   0.76  2.2    0.0652
  24.51   0.81  2.2    0.0662
  25.34    0.86   2.3     0.0667
  26.44    0.89     2.3       0.0678
};


% Linia aproksymacji (czarna, przerywana)
\addplot [smooth, black, dashed] table [
  x=x,
  y={create col/linear regression={y=y}}
]  {
  y  x ey   ex
  6.56    0.27  1.6    0.0554
  10.47    0.36  1.9    0.0572
  13.2    0.44     2.0 0.0588
  15.30    0.5   2.0  0.06
  17.1   0.56     2.1  0.0612
  18.93   0.62  2.1    0.0624
  20.42   0.67  2.1    0.0634
  22.1   0.72   2.1 0.0644
  23.19   0.76  2.2    0.0652
  24.51   0.81  2.2    0.0662
  25.34    0.86   2.3     0.0667
  26.44    0.89     2.3       0.0678
};


\end{axis}
\end{tikzpicture}

\begin{tikzpicture}
\begin{axis}[
  title={Zależność R(I) dla oporu drutowego },
  ylabel={Rezystancja [R]},
  xlabel={Natężenie [mA]},
  grid=major,
  width=1\textwidth,
  height=1\textwidth,
  enlargelimits=false,
  scaled y ticks=false,
  xmin=10,
  xmax=200,
  ymin=110,
  ymax=118,
  mark size=1, % Rozmiar punktów pomiarowych
  only marks,  % Tylko punkty, bez linii
]

% Linie dla Schematu A (dla drutu oporowego)
\addplot+ [
  error bars/.cd,
  x dir=both,
  x explicit,
  y dir=both,
  y explicit,
] table [
  x=x,
  y=y,
  x error plus=ex,
  x error minus=ex,
  y error plus=ey,
  y error minus=ey,
]  {
  y  x ey   ex
  114.36    17.4  3.2    0.3
  114.16    34.6  2.6    0.51
  114.34   51.6  2.4    0.71
  114.38    68.8  2.3   0.92
  114.33   85.8    2.2   1.12
  114.34   103.2 2.2    1.33
  114.28   120.4 2.2    1.54
  114.21  137.9  2.1   1.75
  114.08   154.8  2.1    1.95
  113.98   172.3  2.1    2.16
     113.80 189.0   2.6     2.37

};

% Linia aproksymacji (czarna, przerywana)
\addplot [smooth, black, dashed] table [
  x=x,
  y={create col/linear regression={y=y}}
] {
  y  x ex   ey
  114.36    17.4  3.2    0.3
  114.16    34.6  2.6    0.51
  114.34   51.6  2.4    0.71
  114.38    68.8  2.3   0.92
  114.33   85.8    2.2   1.12
  114.34   103.2 2.2    1.33
  114.28   120.4 2.2    1.54
  114.21  137.9  2.1   1.75
  114.08   154.8  2.1    1.95
  113.98   172.3  2.1    2.16
     113.80 189.0   2.6     2.37

};


\end{axis}
\end{tikzpicture}




\end{document}
